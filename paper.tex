\documentclass[aip, jcp, reprint, onecolumn]{revtex4-2}

\bibliographystyle{apsrev4-2}

\usepackage{physics}
\usepackage{amsmath}
\usepackage{amssymb}
\usepackage{mathtools}
\usepackage{graphicx}
\usepackage{dcolumn}
\usepackage[colorlinks=true, linkcolor=black, urlcolor=blue, citecolor=black, anchorcolor=black]{hyperref}

\graphicspath{{"figures/"}}
\begin{document}
%Title of paper
\title{Coherent IR-Hyper-Raman Four Wave Mixing Vibrational Spectroscopy}


\author{Ryan P. McDonnell} 
\author{Daniel D. Kohler}
\author{John C. Wright} \email{wright@chem.wisc.edu}

\affiliation{Department of Chemistry, 
        University of Wisconsin - Madison, 
        Madison, Wisconsin 53706, 
        United States of America}

\date{\today}

\begin{abstract}
% DDK: change first sentence, we do not care about vibrational coupling here...
Nonlinear, four wave mixing vibrational spectroscopies are often used to probe intramolecular vibrational coupling and relaxation dynamics in isotropic media.
Most of these methods rely on infrared and/or Raman transitions but methods involving hyper-Raman transitions are also possible. 
Nonlinear spectroscopy involving hyper-Raman transitions provides a useful analogue to infrared and Raman based nonlinear spectroscopies to understand the structure and dynamics of isotropic systems.
Hyper-Raman Difference Frequency Generation (HDFG) spectroscopy is an underdeveloped four wave mixing vibrational spectroscopy based on the hyper-Raman transition. 
% The method is similar in implementation to that of vibrational sum frequency generation (vSFG) and difference frequency generation (vDFG) spectroscopy.
Despite several experimental reports on HDFG, its spectroscopic properties have not been fully explored.  
To this end, we investigate the selection rules and behavior of HDFG spectroscopy as an upconverted infrared spectroscopy and as a probe of vibronic coupling in molecular systems.
HDFG shows promise as a method to disentangle vibrational spectra and dynamics in isotropic systems.

\end{abstract}

\maketitle

\section{Introduction}
Coherent multidimensional spectroscopy (CMDS) is a family of nonlinear spectroscopy methods that form the optical analogue of multidimensional nuclear magnetic resonance (NMR) spectroscopy.\cite{Cho2008}
Multiresonant, four wave mixing CMDS experiments, first proposed by Oudar and Shen in 1980,\cite{RN307} directly probe coupling and correlations between different vibrational, electronic, and vibronic states. \cite{RN281, RN103, Cho2008} 
Four wave mixing CMDS has resolved anharmonicities, ultrafast dynamics, and other inter- and intra-molecular couplings in numerous systems. \cite{Cho2008, Gaynor2017, Ziegler2018, Ogilvie2019, Bonn2021, RN325, Chen2024}
To target vibrational and electronic coupling, methods commonly make use of Raman transitions, e.g., doubly vibrationally enhanced spectroscopy (DOVE), triple sum frequency (TSF), coherent anti-Stokes Raman spectroscopy (CARS), resonant impulse stimulated Raman spectroscopy (RISRS).\cite{RN103, Dhar1994, RN345, Cho2000, RN491}
Here we explore the less-utilized hyper-Raman transitions in four wave mixing CMDS.

\begin{figure*}[!htbp]
	\centering
	\includegraphics[width=6.66in]{taxonomy.png}
	\caption{
		Some spectroscopic methods for investigating (top) electronic-vibrational coupling and (bottom) e-v-x (x: e, v, e-v, e-v-e) coupling through different order nonlinear processes.
		The light-matter interactions are depicted using Wave Mixing Energy Level (WMEL) diagrams.\cite{RN286}
		Solid and dashed horizontal lines indicate real and virtual states, whereas solid and dotted arrows indicate ket and bra side transitions, respectively. 
	}
	\label{fig:comparisonwmel}
\end{figure*}

The hyper-Raman transition is the nonlinear, two photon analogue to the Raman transition.\cite{Terhune1965, Cyvin1965, Andrews1978}
Two photons of frequency $\omega_a+\omega_b$ inelastically scatter with matter, promoting or demoting a vibrational mode of frequency $\omega_v$ and emitting a single photon of frequency $\omega_a + \omega_b \mp \omega_v$. 
The single photon emission is a significantly different frequency from the excitation frequencies, making rejection of excitation scatter much easier.
Unlike Raman, all infrared active modes are hyper-Raman active, which allows infrared active modes to be probed through an inelastic scattering event. \cite{Andrews1978}
One drawback is that hyper-Raman transitions are typically weak compared to their Raman counterparts, even when accounting for the high intensities available to ultrafast pulsed lasers.\cite{RN515, Kelley2010}
CMDS methods partially mitigate this issue because emission is spatially coherent so that all emission is directional and easily collected.
In fact, hyper-Raman transitions have been observed in several CMDS studies (sometimes under the moniker of SIVE).\cite{Zilian1994, RN350, RN416, RN351, RN352, RN353, Chen1998, RN362, RN418, Wang2021, Bonn2024, McDonnell2024}

Hyper-Raman difference frequency generation (HDFG) spectroscopy and hyper-Raman sum frequency generation (HSFG) spectroscopy are four wave mixing CMDS methods based on direct excitation of a vibrational mode through IR absorption and subsequent scattering to or from the vibrational mode through a hyper-Raman transition.
In HDFG/HSFG, an infrared pulse is resonant with a vibrational mode, while two other excitations stimulate the hyper-Raman scattering process.
The infrared excitation frequency selects the vibrational modes to be stimulated through the hyper-Raman transition.
The other two excitation frequencies control electronic enhancement for the hyper-Raman scattering event.

\autoref{fig:comparisonwmel} shows both HDFG and HSFG processes and compares them with related spectroscopies that investigate vibrational-electronic coupling.
The three-wave mixing ($\chi^{(2)}$) techniques are often less applicable as they require macroscopic non-centrosymmetry.\cite{RN227}
Hyper-Raman scattering itself is a six-wave mixing ($\chi^{(5)}$) technique; stimulated hyper-Raman will be weak and susceptible to FWM cascades.\cite{RN515, RN243, Cho2000_Cascade}
A coherent six-wave mixing analogue of hyper-Raman scattering, coherent anti-Stokes hyper-Raman spectroscopy (CAHRS), has been proposed and explored theoretically by several authors; however, it is also highly susceptible to FWM cascades.\cite{Berger1978, Bjarnason1980, Cho1997, Cho1998}
Of the many four-wave mixing techniques, HDFG and HSFG have strong similarity to Raman, stimulated Raman, and RISRS in that the electronic state must remove or add a $v$ quantum to the ground state (``ev'' coupling). %rpm: not sure what you're trying to say by this
With the other four-wave mixing techniques (\autoref{fig:comparisonwmel}, second row), the output signal depends on more states than just a single vibrational mode and electronic state (``evx'' coupling). 
As a result, the relationship between the output and vibrational-electronic coupling is more detailed, but also more complicated.
HDFG, HSFG and Raman are more direct probes of electronic-vibrational coupling.
Notably, HDFG and HSFG can be implemented as 2-color experiments (with $\omega_2=\omega_3$);\cite{Cho2001} vibrational sum frequency generation (vSFG) or difference frequency generation (vDFG) setups can be trivially reconfigured to perform HSFG or HDFG measurements, respectively, to investigate ev coupling in bulk systems.

When the hyper-Raman excitation is resonant or near-resonant with electronic states, the brightness of the vibrational feature depends on the nature of the electronic state.
The electronic spectrum can inform on processes that control ultrafast electronic relaxation in molecular and biological systems in similar ways to resonance Raman.\cite{Bredenbeck2015, Arsenault2021}
Since the selection rules of hyper-Raman scattering differ from Raman scattering, the hyper-Raman excitation spectra give a unique alternative to Raman scattering to understand vibrational spectra, electronic structure and vibronic coupling in molecular systems. \cite{Olson2018}

This paper investigates the microscopic parameters that control HDFG output. \cite{Bonn2024, McDonnell2024}
In \autoref{steadystate}, we first identify the vibrational selection rules. 
The selection rules of electronically-resonant HDFG are then identified through a Herzberg-Teller dipole expansion.
HDFG is found to be allowed in any harmonic system that has infrared active vibrations.
A simple harmonic oscillator model system is used to simulate HDFG spectra and illustrate how the excited state potential energy surface affects the hyper-Raman excitation spectrum.
% DDK: potentially need to change this sentence, depending on how latter sections are reworked.
After developing selection rules, the site selective properties, useful experimental aspects of HDFG, and a new scheme to extract the hyper-Raman hyperpolarizability ($\beta_{ijk}$) from HDFG spectra are discussed in \autoref{quant}.
The main findings are summarized in \autoref{conclusion}.


\section{Selection Rules for HDFG Spectroscopy}\label{steadystate}

In this section, we investigate the properties of HDFG and make the connections between HDFG and hyper-Raman scattering explicit.
HSFG will not be discussed, but the application of our treatment to HSFG is straightforward and the selection rules are similar.
We note that HDFG emission can be phase-matched in media with normal dispersion, while HSFG cannot.\cite{RN278}  

Before presenting selection rules, we note that a potential interferant in HDFG spectroscopy are direct and sequential cascades of second-order vDFG and vSFG processes.\cite{RN297, RN302, RN301}
While unimportant in achiral isotropic systems,\cite{Belkin2000} second-order cascades may become important in media where SFG and DFG are allowed, e.g. chiral media, interfaces or non-centrosymmetric media. 
Nevertheless, in the discussion that follows for the rest of this manuscript, we will assume negligible second-order cascades. 

It is useful to expose relationships between nonlinear output, transition dipoles and hyper-Raman hyperpolarizabilities in the driven limit. \cite{Simpson2004}
Under the electric dipole approximation, the $I^\text{th}$ cartesian component of the third order nonlinear output polarization, ${P}^{(3)}_I$, of a four wave mixing process, induced by electric fields $E_J$, $E_K$, and $E_L$ at output frequency $\omega_4=-\omega_1 + \omega_2 + \omega_3$ is written as (using Einstein summation) \cite{RN307}
\begin{equation} \label{polarization}
{P}^{(3)}_I (\omega_4)  = \chi^{(3)}_{IJKL} E_J(\omega_3) E_K(\omega_2) E_L(\omega_1) 
\end{equation}
where $\chi^{(3)}_{IJKL}$ is the $IJKL$ element of the third order electrical susceptibility, a rank four tensor, generally written as
\begin{equation}\label{eq:nfgamma}
	\chi^{(3)}_{IJKL} = NF \langle \gamma_{ijkl} \rangle.
\end{equation}
Here $N$ is a number density and $\gamma_{ijkl}$ is the third-order polarizability (i.e., second hyperpolarizability).
The brackets indicate an orientational average.\cite{Andrews1977}
Uppercase letters refer to laboratory frame coordinates and lower case letters refer to corresponding molecular frame coordinates.
$F=\prod_j \frac{n(\omega_j)^2 + 1}{3}$ is the Lorentz local field factor, where $n$ is the (frequency-dependent) refractive index. 

By propagating density matrix elements in the steady state limit under the rotating wave approximation, the HDFG hyperpolarizability is \cite{RN133}
\begin{equation}\label{sivegamma}
		\gamma_{ijkl}^{vg} =	- \sum_{e,m} \frac{1}{\varepsilon_0} \frac{1}{4D} \frac{1}{\hbar^3} \frac{\mu^{ve}_{i} \mu^{em}_{j} \mu^{mg}_{k} \mu^{gv}_{l} }{\Delta_{ev} \Delta_{mv}\Delta_{gv}}  \rho_{gg}
\end{equation}
% DDK: w_j is not the jth frequency--these terms add with each interaction!  consider indexing delta with w_j
where: $\mu^{ab}_{j}$ is the $j^{th}$ element of $\mel{a}{\vec{\mu}}{b}$, $\Delta_{kl} = \omega_{kl} - \omega_{j} - i\Gamma_{kl}$, $\omega_j$ is the frequency of the j$^{th}$ input field, $\Gamma_{kl}$ is the dephasing of $\rho_{kl}$ and $\rho_{gg}$ is the ground state population.
$D$, the Maker-Terhune degeneracy factor, accounts for permutation symmetry of the excitation fields.\cite{RN134} 
For a HDFG experiment using two (three) distinct, non-degenerate input fields, $D = 3 (6)$.

% DDK: or should this be "Vibrational Selection Rules"?  Does Placzek == nonresonant?
\subsection{Vibrational Selection Rules}

% DDK: moved SIVE mention to intro
It is useful to investigate the selection rules without vibronic resonance to understand how HDFG can be used as an upconverted infrared spectroscopy.
We first investigate the HDFG selection rules through a Placzek type formalism.
When $\omega_2+\omega_3$ is significantly detuned from resonance and the vibronic nature of the virtual states is ignored, the Placzek approximation can be used to study selection rules in terms of the hyper-Raman hyperpolarizability, $\beta$. \cite{Placzek1934, Long1970, Altmann1982}
Summing over the virtual states $\ket{e}$ and $\ket{m}$ forms the hyper-Raman hyperpolarizability so that\cite{Long1970} 
\begin{equation}\label{sivebeta}
	\gamma_{ijkl} =	-\frac{1}{\varepsilon_0} \frac{1}{4D \hbar}\frac{\beta^{vg}_{ijk} \mu^{gv}_{l}}{\Delta_{gv}} \rho_{gg}.
\end{equation}
Since all infrared active transitions are hyper-Raman active, HDFG is allowed for any infrared active transition. \cite{Andrews1978}
This selection rule is generally valid for any HDFG or HSFG process when $\omega_2$ and $\omega_3$ are sufficiently detuned from any resonance.

The general selection rule can be simplified by considering excitation of an infrared active normal mode.  
Assume state $v$ is the $v^{\text{th}}$ normal mode.
We Taylor expand the dipole and first hyperpolarizability operators to first order in the normal mode coordinate $Q_v$ about equilibrium:\cite{Long1970, Shen90}
\begin{subequations}
	\begin{equation}
		\mu_l = \mu_{l,0} + \left(\frac{\partial \mu_l}{\partial Q_v}\right)_0 Q_v 
	\end{equation}
	\begin{equation}
		\beta_{ijk} = \beta_{ijk,0} + \left(\frac{\partial \beta_{ijk}}{\partial Q_v}\right)_0 Q_v
	\end{equation}
\end{subequations}
Substituting into \autoref{sivebeta} gives the HDFG hyperpolarizability to $\order{Q_n}$ as \begin{equation}\label{SIVEselection}
	\gamma_{ijkl} =	-\frac{1}{\varepsilon_0} \frac{1}{8D \omega_{vg}}  \frac{1}{{\Delta_{gv}}} \ \left(\frac{\partial \beta^{vg}_{ijk}}{\partial Q_v}\right)_0 \left({\frac{\partial \mu^{gv}_{l}}{\partial Q_v}}\right)_0  \rho_{gg}
\end{equation}
where we have used $\mel{v}{Q_v}{g} \equiv \mel{1}{Q_v}{0} = \sqrt{\frac{\hbar}{2\omega_{vg}}}$ and $\omega_{vg}$ is the characteristic $\ket{0} \rightarrow \ket{1}$ frequency.\cite{RN459}
Since this expression is non-zero in the harmonic oscillator limit, HDFG output is allowed for normal modes in the harmonic limit. 


\subsection{Alternative Time Orderings}
% DDK: this section uses vibronic labels, but we have not used them yet--is it necessary?
It is useful to briefly discuss the impact of time ordering on HDFG output.
Dependent upon the time-ordering schemes, it is possible to eliminate output through destructive interference.
While we have only isolated one specific HDFG process (\autoref{fig:comparisonwmel}), there are several other pathways which can provide HDFG output. \cite{RN352}
If $\omega_2$ ($\omega_1$) becomes resonant (non-resonant), where $\omega_2 > \omega_1$, then other HDFG pathways appear (\autoref{fig:sivewmel2}).\cite{McDonnell2024} 
While the methods in \autoref{fig:sivewmel2} possess identical selection rules to \autoref{fig:comparisonwmel}, the pathways presented in \autoref{fig:sivewmel2} will interfere, as they have oppositely signed amplitudes.
\begin{figure*}[!htbp]
	\centering
	\includegraphics[width=6.66 in]{figures/timeorderedwmel.png}
	\caption{WMEL diagrams of HDFG pathways for when (a,b) $\omega_1$ (black arrow) interacts first and (c-f) $\omega_2$ (orange arrow) interacts first. 
		The states are labeled as $\ket{a,b}$, where $a$ refers to an electronic state and $b$ refers to quanta in a vibration on $a$.
		In all cases, $\omega_2 > \omega_1$.
	}
	\label{fig:sivewmel2}
\end{figure*}

In the limit where $\omega_2+\omega_3$ are detuned from an electronic resonance, it becomes simple to interpret the interference schemes. 
The case where $\omega_1$ interacts first and is resonant, and $\omega_2 + \omega_3$ are nonresonant, yields two contributed pathways (a) and (b).
These pathways possess identical phase and transition moments, which yields total constructive interference.
Conversely, diagrams (c,d) and (e,f), when detuned from an electronic resonance, have identical transition moments but opposite phase. 
As a result, the output from (c,d) and (e,f) completely destructively interfere; no signal can be obtained from these pathways when significantly detuned. \cite{RN287, McDonnell2024}

When $\omega_2 + \omega_3$ become electronically resonant, the interference changes for the \autoref{fig:sivewmel2}c-f pathways.
Note that only pathways (c,e,f) possess identical transition moments.
As a result, pathways (c,d) survive and can provide output when $\omega_2+\omega_3$ is resonant with an electronic state. 
These results provide a simple rule of thumb for collecting HDFG spectra: use \autoref{fig:sivewmel2}a,b pathways to measure spectra when significantly detuned from an electronic resonance, and any pathway can measure spectra when electronically resonant.
However, since electronic states commonly have short dephasing times and have significant inhomogeneous broadening,\cite{Dong2015, Lewis2015} it is likely the \autoref{fig:sivewmel2}f pathway will become triply resonant and possibly contribute greater than the other pathways. 
Such instances are system dependent. 
For consistency, discussion will be isolated to the \autoref{fig:sivewmel2}a pathway in the sections that follow.

\subsection{Electronically resonant and pre-resonant: Vibronic Selection Rules}
% TODO: note that the IR dipole selection rule persists here--this is all about w2+w3 resonance

While the Placzek treatment provides the general source of HDFG output, it does not predict the behavior of $\beta_{ijk}$ as $\omega_2 + \omega_3$ is changed.
We now explicitly consider electronic resonances through the $A,B,C$ decomposition of $\beta$ introduced by Chung and Ziegler, analogous to those found in Albrecht's treatment of Raman spectroscopy.\cite{Albrecht1961, Ziegler1988} 
To employ this formalism, we write the states in a Born-Oppenheimer basis $\ket{a,b}$, where $\ket{a,b} = |a(Q)) \otimes \ket{b}$ for electronic states $\{|a(Q))\}$ and vibrational states $\{\ket{b}\}$ (i.e., adiabatic approximation). \cite{BornOppenheimer, Tang1970}
We will henceforth suppress the dependence of the electronic states on $Q$ for simplicity.

For consistency with previous reports, we use $\vec{R}$ to denote electric transition dipole moments; $\vec{\mu}$ is reserved for transitions on the ground electronic state. \cite{Tang1970}
Following the approach which gave \autoref{sivegamma} and using the time ordering and state labeling of \autoref{fig:sivewmel2}a, we find
\begin{equation}\label{drgamma_notaylor}
	\gamma_{ijkl} = -\frac{1}{\varepsilon_0} \frac{1}{4D \hbar^3} \sum_{m,n,e,v'} \frac{
		R_{i}^{gv, ev'} 
		R_{j}^{ev',mn} 
		R_{k}^{mn,g0} 
		R_{l}^{g0,gv} 
	}{\Delta_{g0,gv}
		\Delta_{ev', mn}
		\Delta_{mn, g0}
	}
\end{equation}
where $R_{i}^{ab,cd}$ is the i$^{th}$ element of $\mel{a,b}{\vec{R}}{c,d}$.
It is common in the hyper-Raman community to write $\Delta_{ev', mn} \Delta_{mn, g0} = \Delta_{ev', g0}$.
Using our definition of vibronic states, we can write, for example,
$R_{i}^{gv,ev'} = \mel{v}{M_i^{ge}}{v'}$, where $\vec{M}_{ab} = (a|\vec{R}|b)$.\cite{Ziegler1974}
Expanding $\vec{M}^{ij}$ to $\order{Q}$ about the equilibrium point of the ground state potential surface as
$\vec{M}^{ij} = \vec{M}^{ij}_0 + \sum_z \frac{\partial\vec{M}^{ij}}{\partial Q_z} Q_z$
yields $A, B, C$ coefficients similar to those in the Albrecht formalism of Raman scattering, \cite{Albrecht1961, Warshel1977, Ziegler1988} so that
\begin{equation}
		\gamma_{ijkl} \sim \left(A_{ijk} + B_{ijk} + C_{ijk}\right) \frac{\mel{v}{\mu_{l}}{0}} {\Delta_{g0,gv}}
\end{equation}
The $A$ term contains the static ($\order{Q^0}$) transitions (i.e., Condon approximation), the $B$ term depends upon $\order{Q}$ transitions (Herzberg-Teller contributions), and the $C$ term depends on $\order{Q^2}$ transitions. 
The $C$ term is suppressed in the following discussion as it depends on one and two photon forbidden transitions, making its contribution to $\gamma_{ijkl}$ roughly two orders of magnitude lesser than $A_{ijk}$. \cite{Ziegler1988, Neddersen1989, Bonang1992}
Note that some reports obtain $A, B$ coefficients through a Herzberg-Teller expansion of the electronic states to expose vibronic couplings via $\partial H / \partial Q$, where $H$ is the electronic Hamiltonian.\cite{HerzbergTeller1933, Petrov1985, Neddersen1989, Baranov1990}
This approach is not used here as the expansion of $\vec{M}^{ij}$ in normal mode coordinates provides sufficient physical insight into hyper-Raman selection rules. 

% By contracting over the virtual vibrational states $\ket{n}$, 
The $A$ and $B$ coefficients, where $B = B_1 + B_2$, are written as:
\begin{widetext}
\begin{subequations}\label{ABterms}
\begin{equation}
	\begin{split}
		A_{ijk} = \frac{1}{\hbar^2}\sum_{m,e,v'} M^{ge}_{0,i} 
		M^{em}_{0,j} 
		M^{mg}_{0,k}
		 \langle v | v' \rangle
		 \langle v' | 0 \rangle 
		 \frac{1}{\Delta_{ev', g0}}
		 \\
	\end{split}
\end{equation}
	\begin{equation}
		\begin{split}
			B_{1_{ijk}} &= \frac{1}{\hbar^2} \sum_{m,e,v',z} M^{ge}_{0,i} \left(
				\frac{\partial M^{em}_{j}}{\partial Q_z} M^{mg}_{0,k}  
				+M^{em}_{0,j} \frac{\partial M^{mg}_{k}}{\partial Q_z}
			\right)
			\langle v | v' \rangle \mel{v'}{Q_z}{0} \frac{1}{\Delta_{ev', g0}}\\
		\end{split}
	\end{equation}
	\begin{equation}
	\begin{split}
			B_{2_{ijk}} = \frac{1}{\hbar^2} \sum_{m,e,v',z} \frac{\partial M^{ge}_{i}}{\partial Q_z} M^{em}_{0,j} 
			M^{mg}_{0,k} \mel{v}{Q_z}{v'} 
			\langle v' | 0 \rangle 
			\frac{1}{\Delta_{ev', g0}}
	\end{split}
	\end{equation}
\end{subequations}
\end{widetext}
where $\langle a | b \rangle$ and $\mel{a}{Q}{b}$ are Franck-Condon factors and Herzberg-Teller-type integrals, respectively. 
At this point, the expressions are valid for both electronically resonant and non-resonant cases.

For the case of electronic resonance, further simplifications can be made.
We restrict consideration to electronic transitions between $|g)$ and $|e)$.  
Note that HDFG experiments selectively excite quanta in ground state vibrational modes. 
By working only in terms of a single normal mode and assuming the infrared pulse selectively excites $\ket{g,0} \rightarrow \ket{g,1}$ (i.e., removing the sum over $z$ and taking $v=1$), 
% $v' = \{0,1,2\}$, 
the A and B terms are rewritten as 
	\begin{subequations}\label{ABterms_DR}
		\begin{equation}
			\begin{split}
				A_{ijk} = \frac{1}{\hbar}\sum_{v'} M^{ge}_{0,i} 
				\Lambda^{eg}_{0,jk}
				\langle 1 | v' \rangle
				\langle v' | 0 \rangle 
				\frac{1}{\Delta_{ev',g0}}
				\\
			\end{split}
		\end{equation}
		\begin{equation}
			\begin{split}
				B_{1_{ijk}} &= \frac{1}{\hbar} \sum_{v'} M^{ge}_{0,i} \langle 1 | v' \rangle 
				\frac{\partial \Lambda^{eg}_{jk}}{\partial Q} \mel{v'}{Q}{0} 
				\frac{1}{\Delta_{ev', g0}}\\
			\end{split}
		\end{equation}
		\begin{equation}
			\begin{split}
				B_{2_{ijk}} = \frac{1}{\hbar} \sum_{v'} \frac{\partial M^{ge}_{i}}{\partial Q} 
				\Lambda^{eg}_{0,jk} 
				\mel{1}{Q}{v'} 
				\langle v' | 0 \rangle 
				\frac{1}{\Delta_{ev',g0}}
			\end{split}
		\end{equation}
	\end{subequations}
where $\Lambda^{eg}_{0,jk}$ is the electronic two photon transition moment between $g$ and $e$.\cite{McClain1977}
$\Lambda$ contains the summation over $m$ in \autoref{ABterms} (N.B., here we use $\Lambda^{ab}_{ij} = 1/\hbar \sum_m M_i^{am}M_j^{mb} $).

\autoref{ABterms_DR} shows important selection rules for the vibronic transitions.
The $A$ term is allowed whenever an electronic transition has a one- and two- photon transition moment.
The brightness is controlled by vibrational overlap (Franck-Condon factors).
For centrosymmetric molecules, however, an electronic transition cannot be both one- and two- photon allowed, so the $A$ term will vanish.\cite{Milojevich2013}
This makes HDFG a unique tool for studying the electronic structure of centrosymmetric species in isotropic media, as it is sensitive to non-Condon effects.\cite{Olson2018}
In the absence of an $A$ term, the $B$ terms will dominate.
The $B_1$ and $B_2$ terms involve different types of vibronic coupling pathways.
In $B_1$, the Herzberg-Teller coupling is associated with the two-photon absorption process, whereas in $B_2$, the Herzberg-Teller coupling is associated with one-photon emission.

\begin{figure*}[!htbp]
	\centering
	\includegraphics[width=6.66in]{drsive_spectrum.png}
	\caption{Contributions of $A, B$ (\autoref{ABterms_DR}) to HDFG spectrum for a simple two harmonic well system.
		(a) Potential energy surfaces for a two-well system, (b) 1D HDFG spectrum for $\xi = 0.5$, $\omega_1 = \omega_{g1, g0}$. 
		It is assumed that $\omega_2 = \omega_3$.
		Note that (b) plots magnitudes of the labeled quantities.
		The vibrational states on the $|g)$ and $|e)$ manifolds are spaced 2200 cm$^{-1}$ apart, with the vibronic states $\ket{e,v'}$ given linewidths of 700 cm$^{-1}$ and $\hbar \omega_{eg}$ = 30000 cm$^{-1}$.
		Dotted lines in (b) denote the $v'$ = 0, 1, 2 vibronic resonances. 
		The vibronic one and two photon absorption operators are scaled such that $\abs{B}/\abs{A} \sim$ 0.1, following Chung and Ziegler. \cite{Ziegler1988}}
	\label{fig:doubres_spec}
\end{figure*}

The impact of vibronic coupling in the 2D spectrum is made clear by using a simple model. \cite{Kundu2022}
Using a two-well system described by 
\begin{subequations}
	\begin{equation}
		H_g = \frac{p^2}{2m} + \frac{1}{2} \hbar \omega q^2
	\end{equation}
and
	\begin{equation}
		H_e = \frac{p^2}{2m} + \frac{1}{2} \hbar \omega (q-\xi)^2 +\hbar \omega_{eg}
	\end{equation}
\end{subequations}
for the ground ($H_g$) and first excited ($H_e$) states, where $p$ is the momentum of the normal mode, the A and B contributions can be evaluated using Franck-Condon and Herzberg-Teller integrals tabulated in terms of the dimensionless normal mode $q$ and offset parameter $\xi$ (\autoref{fig:doubres_spec}a). \cite{Carlson1988thesis} 
Note that $q = \sqrt{\frac{m^2\omega}{\hbar}} Q$.
The spectrum is simulated for an electronic state which is both one and two photon allowed.
A simulated spectrum (\autoref{fig:doubres_spec}b) highlighting the contributions of these terms on potential wells with $\xi = 0.5$ dissects the $A$ and $B$ contributions to the HDFG spectrum.
This stimulated hyper-Raman type spectrum would be the expected result in a HDFG experiment when investigating the vibronic structure of the electronic state coupled to $\ket{g,1}$.

Simulating the HDFG spectrum for the simple two-well system allows dissection of selection rules and spectral signatures without the need for complex couplings.
The spectrum selectively probes coupling between the vibronic states and $\ket{g,0}$, $\ket{g,1}$.
There are two relevant regions of interest: the resonant region (2$\omega_2$ $\in$ [29000, 35000] cm$^{-1}$) and the non-resonant region.
Focusing far away from the resonant region ($2\omega_2 \sim$ 20000 cm$^{-1}$), the nonresonant $A$ term response is roughly equivalent to that of the resonant $B$ term response. 
Previous reports suggest that $A$ term contributions may become negligible when significantly detuned from a resonance such that $\Delta_{ev', g0} \approx \Delta_{eg}$, i.e., the resonance denominator loses $\ket{v'}$ dependence. \cite{Neddersen1989}
The lack of $\ket{v'}$ dependence in the resonance denominator would cause $A$ to vanish through closure ($A_{ijk} \sim \sum_{v'} \langle 1|v' \rangle \langle v'|0\rangle$ = $\delta_{10} = 0$).
However, the simulated spectrum (\autoref{fig:doubres_spec}b) shows a non-negligible $A$ contribution to HDFG output well beyond $\sim$ 20000 cm$^{-1}$, the limit for this model system where $\abs{\omega_{ev',g0} - 2\omega_2} \gg 2\Gamma_{ev',g0}$.
In other words, it appears difficult to escape the pre-resonance regime of hyper-Raman scattering.
As such, it is difficult to eliminate $A$ term resonances in HDFG spectra, even when largely detuned ($\gg 2\Gamma_{ev',g0}$) from the electronic state. 
This is not unexpected, as when $\abs{\omega_{ev',g0} - 2\omega_2} \gg 2\Gamma_{ev',g0}$, $A_{ijk} \sim \left(2\omega_2 - \omega_{ev',g0}\right)^{-1}$, which slowly converges as $\omega_2 \rightarrow 0$.
Similar ideas have been seen in spontaneous Raman scattering. \cite{Warshel1977, Li1990, Gong2015}
As such, it may prove difficult to examine non-Condon effects ($B$ term contributions) in HDFG spectra of noncentrosymmetric systems. 

In the resonant region, the $A$ term contains character from all three possible transitions between $\ket{g,0}$, $\ket{e,v'}$ and $\ket{g,1}$. 
Significant contributions from $\ket{e,0}$ and $\ket{e,1}$ in the spectrum correspond well with expectations given the relative wavefunction overlap in \autoref{fig:doubres_spec}a. 
The small contribution to $A$ involving transitions with $\ket{e,2}$ is consistent with the large change in quantum number from $\ket{g,0}$ and its considerable displacement from the equilibrium point of $|g)$.
The dependence of $A$ on these Franck-Condon factors is in agreement with expectations following \autoref{ABterms_DR}. 

Unlike $A$ term contributions, the $B_1$ and $B_2$ terms are dependent upon both Franck-Condon and Herzberg-Teller contributions.
The $B_1$ term, whose vibronic coupling term is involved in the two-photon absorption event from $|g)$ to $|e)$, seemingly has minimal $0-0$ contributions. 
This is expected, as the Herzberg-Teller integral $\mel{0}{Q}{0}$ is smaller than $\mel{1}{Q}{0}$ and $\mel{2}{Q}{0}$. 
An explanation for the contribution from the $0-1$ transition in $B_2$ is similar.
Vibronic contributions to $B_2$ arise from the one photon emission transition; $\mel{1}{Q}{1}$ is smaller than the $\mel{1}{Q}{0}$ and $\mel{1}{Q}{2}$ contributions to the $B_2$ term.
However, the $0-2$ transition is smaller in amplitude than the $0-1$ transition because it is also dependent upon a $\langle 2 | 0 \rangle$ Franck-Condon factor.
In a realistic molecular system, where higher order contributions (e.g. Duschinsky rotation) affect vibronic coupling signatures, the spectral signatures will change to incorporate anharmonicites. \cite{Duschinsky1937, Carlson1990, Kundu2022}
Investigating coupling between $\ket{g,2}$ and the same $\ket{e,v'}$ vibronic states will likely assist in dissecting the site-selective electronic spectra.

\section{Comparisons with Other Spectroscopies}\label{quant}
Unlike resonant hyper-Raman spectroscopy, which allows transitions between different normal modes of varying quanta, the infrared pulse in the HDFG provides a type of site selectivity,\cite{RN103, Carlson1991} so that only the mode pumped by $\omega_1$ can be involved in vibronic coupling. 
While a typical hyper-Raman experiment can probe similar information by tracking the relative intensity of a mode with different excitation color, hyper-Raman output is incoherent and difficult to detect. \cite{Kelley2010}
Since HDFG output depends upon stimulated hyper-Raman scattering, the output is highly directional, increasing the efficency of detecting hyper-Raman based output. 
As a result, it is easier to assess changes in intensities and quantify coupling strengths in terms of $\chi^{(3)}$ (\textit{vide infra}).

HDFG methods can supplement other coherent Raman based methods as a method to investigate vibronic coupling mechanisms between a specific vibration on the ground state and vibrational states on an arbitrary excited state.\cite{RN103}
The combined infrared and hyper-Raman properties of HDFG make it highly sensitive to molecular electronic structure.
Particularly in the case of a centrosymmetric species, non-Condon effects dominate in the spectrum as the $A$ term vanishes from the presence of an inversion center.
The site selective properties of HDFG can allow for parsing of non-Condon effects in different vibrational modes, and extend the use of CMDS to understand the electronic and vibronic structure of simple and complex species.

The hybrid infrared / hyper-Raman properties of HDFG make it an intriguing analogue of resonance IR spectroscopy.
First demonstrated by Boyle et al., resonance IR uses a $2\vec{k}_1 + \vec{k}_3$ pathway, identical to the TSF pathway in \autoref{fig:comparisonwmel}, to amplify weak infrared modes through vibronic coupling, dependent upon the Raman $A,B,C$ coefficients. \cite{RN491}
However, the $2\vec{k}_1 + \vec{k}_3$ resonance IR method implicitly depends upon the coupling of an overtone $\ket{g,2v}$ to vibronics $\ket{e,v'}$.
% DDK: did we not already discuss this?  Why is this not the introduction? %rpm: I see this as a place to expand on it since discussion in intro is succinct
Alternatively, HDFG provides another form of resonance IR spectroscopy, but only requires coupling between $\ket{g,v}$ and $\ket{e,v'}$.
This form of resonance IR spectroscopy would also uniquely depend upon one and two-photon absorption components, as discussed in terms of the hyper-Raman $A,B$ coefficients.
By combining resonance IR in the HDFG and $2\vec{k}_1 + \vec{k}_3$ geometries, the sensitivity of the Raman and hyper-Raman $A,B$ coefficients to ground state vibrations and electronic structure can be assessed. 
Such experiments can help assess the quality of computationally calculated ground and excited state potential energy surfaces. 

Despite their increasing use for dissecting intra and intermolecular interactions, four wave mixing techniques are usually not feasible for most laboratories, as they usually demand the use of multiple optical parametric amplifiers and/or complex acousto-optic modulators. \cite{Chen2016}
Unlike most four wave mixing techniques, HDFG can be performed using only two input beams (i.e., two color HDFG).
In the case of two color HDFG, only one tunable infrared optical parametric amplifier is needed, as the oscillator which pumps the infrared OPA can be used to perform the two-photon upconversion needed to generate HDFG output.
Vibrational sum frequency generation (vSFG) is a $\chi^{(2)}$ technique whose output, when only vibrationally resonant, scales as $\chi^{(2)}_{IJK} \sim \langle \alpha_{ij} \mu_k \rangle$, where $\alpha_{ij}$ is the Raman polarizability tensor, whose phase-matching constraint is  $\vec{k}_3 = \vec{k}_1 + \vec{k}_2$.
HDFG is phase-matchable under $\vec{k}_4 = -\vec{k}_1 + 2\vec{k}_2$ conditions for most, if not all, vibrational resonances.
Laboratories which use vSFG to investigate interfacial species at buried interfaces thus have the ingredients necessary to perform a HDFG experiment in a transmission geometry. \cite{Piontek2023_1}
This would allow vSFG practitioners to perform measurements in the bulk, such as measuring steady-state vibrational spectra, free induction decay and vibrational population lifetimes through HDFG or pump-HDFG-probe.
 
To demonstrate the feasibility of HDFG for vSFG practitioners, we compare the relative output of each process. 
Macroscopically, oddness in spatial inversion of the vSFG polarization eliminates output from centrosymmetric species under the electric dipole approximation.\cite{RN132, RN133}
As a result, vSFG output is significantly reduced relative to most third order spectroscopies because it depends on surface number density, many orders of magnitude smaller than the bulk surface density, which is weak even for even for non-linear cross-sections.
However, HDFG depends upon hyper-Raman cross-sections, which are commonly several orders of magnitude weaker than corresponding Raman transitions.\cite{RN515}
As calculated in \autoref{appendixA}, these effects cancel, and the output polarization of HDFG and vSFG are roughly identical ($\abs{P_\text{vSFG} / P_\text{HDFG}} \sim 1$).
Transient HDFG, or IR-pump-HDFG-probe, analogous to IR-pump-vSFG-probe, is therefore also feasible under similar experimental conditions, and provides vSFG practitioners a simple method to measure bulk dynamics without significant changes in an experimental setup. 
Therefore, HDFG is a feasible spectroscopy for practitioners of vSFG both in terms of experimental setup and detection feasibility.

With the selection rules of HDFG understood for vibrational spectroscopy, and with HDFG generating a large enough output polarization, it becomes possible to obtain quantitative information from its spectra.
Lineshape analysis is essential for extracting quantitative information from CMDS spectra.
Scanning across resonances create dispersive lineshapes, i.e, self-heterodyning, which inform on $\Re(\chi^{(3)})$ and $\Im(\chi^{(3)})$.\cite{Levenson1974_1, Levenson1974_2}
In the method introduced by Levenson and Bloembergen (Bloembergen interferometry experiment), an internal standard interferes with the resonant lineshape.
Self-heterodyning of the internal standard signal measures $\chi^{(3)}$ and does not require measurement of absolute intensities. 
Resonant lineshapes are also complicated by amplitude level interference between the sample,  substrate and/or sample cell windows, which must be accounted for to obtain quantitatively correct $\chi^{(3)}$ values. \cite{RN362, RN418}
Most quantitative methods in an n$^{th}$ order CMDS experiment are used to measure $\chi^{(n)}$ values to compare the relative strength of nonlinear processes in different media. \cite{Zhu87, RN351, RN345}
The recorded $\chi^{(n)}$ values provide insight into how microscopic quantities ($\vec{\mu}, \alpha_{ij}, \beta_{ijk}$) impact nonlinear output.
These quantities are usually measured using their incoherent analogues (IR spectroscopy, spontaneous Raman spectroscopy, spontaneous hyper-Raman spectroscopy). \cite{Levenson1974_2, RN412, Shoute2005}

Compared to $\vec{\mu}$ and $\alpha_{ij}$, it is difficult to calculate $\beta_{ijk}$ values from spontaneous hyper-Raman experiments. \cite{Kelley2010}
Only a few experimental determinations of $\beta_{ijk}$ for vibrational modes have been performed. \cite{Xu1997, Shoute2005, Kelley2010}
Methods reported in the literature to measure absolute hyper-Raman polarizabilities depend upon external standards such as hyperpolarizabilities of dissolved samples or two-photon absorption cross sections, also difficult to measure. \cite{Okuno2020}
Since the Bloembergen interferometry experiment only relies on the third order susceptibility for measured species (e.g., benzene or CaF$_2$),\cite{Levenson1974_2} it is possible to use quantitative four wave mixing spectroscopy to calculate $\beta_{ijk}$ values.
It is thus useful to investigate how a treatment of orientational averaging can extract $\beta_{ijk}$ from the HDFG $\chi^{(3)}_{IJKL}$ expression.

Using \autoref{eq:nfgamma} and \autoref{sivebeta}, we can write
\begin{equation}\label{chi3}
		\chi^{(3)}_{IJKL} = -\frac{NF}{4D \hbar \varepsilon_0 \Delta_{gv}} \langle \beta_{ijk} \mu_l \rangle \rho_{gg}\\
\end{equation}
For simplicity, we take $\rho_{gg} = 1$.
The steps behind orientational averaging of $\gamma_{ijkl}$, a rank four tensor in the molecular frame, are detailed elsewhere.\cite{Andrews1977, McDonnell2024}
Briefly, a tensor in the molecular frame, A$_{ijkl}$, is transformed into an element of the same tensor in the laboratory frame, $A_{IJKL}$, through $A_{IJKL}$ = $\theta^{ijkl}_{IJKL} A_{ijkl} = \langle A_{ijkl} \rangle$, where summation over repeated indices is implied and $\theta$ is the transformation operator. \cite{McDonnell2024}
Orientational averaging shows specific polarization schemes isolate linear combinations of different $\beta_{ijk}$ terms. \cite{Bersohn1966, Willetts1992, Kauranen1996}
By using the expansions of $\beta_{ijk}$ and $\mu_{l}$ to $\order{Q_v}$ found earlier, and knowing that $\Re(\chi^{(3)})$ vanishes when resonant, we see
\begin{equation}\label{betasive}
	\left\langle \frac{\partial \beta_{ijk}}{\partial Q_v} {\frac{\partial \mu_l}{\partial Q_v}} \right\rangle = -\frac{8D \varepsilon_0}{NF}  {\Gamma_{gv} \omega_{vg}} {\Im(\chi^{(3)}_{IJKL})}
\end{equation}
Since $\abs{\partial \mu / \partial Q}$ values can be extracted from FT-IR spectra,\cite{RN459} HDFG can give quantitative information on the magnitude of $\beta_{ijk}$ for infrared active vibrations.
Additionally, by using non-degenerate input frequencies to stimulate the hyper-Raman transition, the asymmetric properties of $\beta_{ijk}$ can be examined and quantified. \cite{Christie1971, Denisov1986, Kozich2007}
These quantitative aspects of HDFG could prove useful for examining how well computational methods calculate $\beta_{ijk}$ values.

\section{Conclusions}\label{conclusion}
Coherent vibrational, hyper-Raman coherent four wave mixing spectroscopies are identified and discussed.
Hyper-Raman difference frequency generation (HDFG) processes, a coherent analogue of infrared spectroscopy, are shown to be non-zero for all harmonically allowed infrared transitions.
We find that HDFG is always allowed for harmonic transitions, making it a potential tool for measuring single quantum coherence lifetimes. 
Through an examination of the hyper-Raman $A,B,C$ terms, the impact of electronic resonance and vibronic coupling in HDFG spectrum was examined.
HDFG possesses types of site-selective properties analogous to those of nanosecond coherent Raman experiments. 
Site-selective properties should allow for a thorough analysis of vibronic coupling schemes in isotropic systems.
Due to the triviality of performing HDFG as a two-beam experiment, HDFG is shown to be feasible for practioners of vibrational sum-frequency generation (vSFG) spectroscopy. 
We show that HDFG can extract the hyper-Raman hyperpolarizability without a need for technically difficult, analytically rigorous hyper-Raman scattering experiments. 
The HDFG method shows promise as a spectroscopic probe of unique electronic structure effects in a variety of molecular systems, but also for understanding the dynamics of single quantum coherences. 

\section{Data Availability Statement}
The workup scripts that support this study are permissively licensed and available for reuse at \href{https://osf.io/2amkq/}{https://osf.io/2amkq/}.

\section{Acknowledgments}
This work received support from the National Science Foundation (Grant no. CHE-2203290).
R.P.M. acknowledges support from the NSF Graduate Research Fellowship Program (Grant no. DGE-2137424). 

\section{Appendix: Comparison of vSFG and two-beam HDFG output}\label{appendixA}
 To motivate the application of HDFG spectroscopy, we perform a calculation to compare two-color HDFG and vSFG output, where absorption effects are neglected for simplicity. \cite{Carlson1989}
 The output polarizations under the electric dipole approximation are written as
	 	\begin{subequations}
		 	\begin{equation}\label{PSFG}
		 		\begin{split}
		 		\abs{P^{(2)}_{\text{vSFG}}} &= \frac{N_\text{surf}}{\ell} \abs{F_{\text{SFG}} \langle \alpha_{ij}\mu_{k} \rangle E(\omega_2)E(\omega_1)} 
		 		\end{split}
			 \end{equation}
		 	\begin{equation}
		 		\begin{split}
			 		\abs{P^{(3)}_{\text{HDFG}}} &= N_\text{bulk} \abs{F_{\text{HDFG}} \langle \beta_{ijk} \mu_{l} \rangle E{(\omega_2)}E(\omega_2)E(-\omega_1)}
		 		\end{split}
			 \end{equation}
		 \end{subequations}
 where N$_\text{bulk}$ is the bulk number density ($\sim$ 10$^{28}$ m$^{-3}$ for H$_2$O$_{(l)}$ at room temperature), and N$_\text{surf}$ is the surface number density ($\sim$ 10$^{16}$ m$^{-2}$ for H$_2$O adsorbed at quartz).\cite{Du1994}	
 For simplicity, we assume that $\chi^{(2)}$ is a spatial constant so that $\int_{0^-}^\ell \mathrm{d}z \langle \alpha_{ij}\mu_{k} \rangle \approx \ell \langle \alpha_{ij}\mu_{k} \rangle$, where $z$ is oriented along the surface normal and $0^{-}$ is the edge of the interface where substrate begins. 
 We take the film thickness ($\ell$) probed by vSFG to be 1 nm, much less than the vSFG coherence length.\cite{RN133}
 A normal dispersion curve is assumed so that $n(\omega_1+\omega_2) \approx n(2\omega_2-\omega_1)$.
 To simplify analysis, orientational averaging is ignored (e.g., $\langle \beta_{ijk} \mu_{l} \rangle \approx \beta \mu$) and the input and output fields are taken to be co-polarized, so that
 \begin{equation}
	 		P_\text{ratio} \equiv \frac{\abs{P^{(3)}_{\text{HDFG}}}}{\abs{P^{(2)}_{\text{vSFG}}}} \approx \frac{N_\text{bulk}}{N_\text{surf}} \abs{\frac{\left(n(\omega_2)\right)^2 + 1}{3}} \frac{\beta}{\alpha} E(\omega_2) \ell \sim 10^3 \frac{\beta}{\alpha} E(\omega_2)\\
	 \end{equation}
 Ziegler has noted that for a field with intensity 10 GW/cm$^{2}$, $\frac{\beta E}{\alpha} \sim 10^{-3} $ for vibrational modes when $E(\omega_2)$ is largely detuned from electronic resonances. \cite{RN515}
 Such an intensity is easily obtained using modern ultrafast sources.
 In this limit, $P_\text{ratio} \sim 1$.
 Since the intensity ratio scales as $\abs{P_\text{ratio}}^2$, we see that the HDFG output is roughly as strong as vSFG, assuming only interfacial contributions in vSFG.
 It is important to note that this calculation assumed negligible dipole and Raman polarizability dependence upon the distance from surface normal, ignored orientational averaging effects, and presumed equivalence of local field factors, all of which can significantly reduce output from either process. 
 Nevertheless, since HDFG produces a number of photons similar to that of vSFG, HDFG is a viable technique for investigating bulk systems.
 This method provides practitioners of sum frequency generation the ability to probe isotropic dephasing dynamics without many meaningful changes in their optical setup, and thus compare  dephasing in the bulk and at the interface.\cite{RN224}

\section{References}
% Create the reference section using BibTeX:
\bibliography{library.bib}

\end{document}
%


